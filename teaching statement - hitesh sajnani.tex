
%\documentstyle[12pt]{article}
%\pdfminorversion=4
\documentclass[a4paper]{article}

\setlength{\oddsidemargin}{0in}
\setlength{\evensidemargin}{0in}
\setlength{\textwidth}{6.5in}
\setlength{\topmargin}{-.3in}
\setlength{\textheight}{9in}

\usepackage{hyperref}
\usepackage{lastpage}
%\pagestyle{empty}
\usepackage{fancyhdr}
\usepackage{pagecounting}
\usepackage[dvips]{color}
\definecolor{gray}{rgb}{0.4,0.4,0.4}
\thispagestyle{fancy}
\fancyfoot[C]{\small \textcolor{black}{Hitesh Sajnani -- }\thepage{}~\small{of}~\pageref{LastPage} -- \small{Teaching Statement}} 
\usepackage{graphicx}
\usepackage{svg}
\usepackage{csquotes}
%\graphicspath{ {./images/} }
\begin{document}
%\usepackage{svg}

\begin{center}
{\LARGE \textbf{Teaching Statement}} \\[.3in]
{\large  \textbf{Hitesh Sajnani}} \\
{\small hsajnani@uci.edu}
\end{center}
\pagestyle{fancy}
\lhead{\textcolor{black}{\it Hitesh Sajnani}}
\rhead{\textcolor{black}{\thepage/\pageref{LastPage}}}


%\vspace*{.5in}

\section*{Why I want to teach?}
I have always thought of teaching to be the most satisfying aspect of an academic career. Looking back, the most influential people in my life have been my teachers. 
They gave me the tools that'd help me grow in my life. The joy in their eyes when they see me grow and the realization that they were part of the process that got me where I am today is absolutely priceless.  
There is nothing more rewarding than knowing and seeing the evidence that you have made an impact on someone’s life. This reward on its own drives me to be an educator. 

I believe that teaching would give me the flexibility to create and experiment with my own environments, a stage to insipre others, and a platform for improving my own skills.  
Moreover, in my opinion, it is very difficult to have a boring day at work when you are a teacher. You always leave work with at least one exciting, funny, or interesting story to tell!

\section*{What experience do I have in teaching?}
\textbf{Teaching:} It was my passion for teaching that originally compelled me to pursue a Ph.D. in the first
place. In my first year of graduate school I was the Teaching Assistant (TA) for the undergraduate
\enquote{Human Computer Interaction} course at University of California, Irvine, which involved a two hour lab session
every week. The professor of this course had an outline for the lab, but it was up to the TA to teach it however they pleased. Over the
duration of this course,  I learnt how to guide students to reach their objectives in class, as every student had different levels of experience in software development. 

I was also a TA for an undergraduate course \enquote{Usability Engineering and Software Development} --- a capstone project course that empower students to tackle real-world design challenges.
It was a great learning experience in guiding the students on their projects and helping them hands-on in developing mobile apps and other software that have a critical impact on audiences across Orange County and beyond. 
In some cases, students’ apps got deployed through the Google or Apple app stores or were put to actual use in the organizations that sponsored the projects, giving the students an incredible advantage going out into the job market.
But more importantly, to this day I meet students who took that course and are now pursuing successful software careers. It is rewarding to know that what I taught in class helped them in their professional life. 

Later I became the TA of \enquote{Introduction to Software Engineering} course. This was specifically interesting as it complemented the research I was pursuing. The syllabus went beyond traditional concepts, methods, and  practice of software engineering and included various state-of-the-art research studies in the area of software engineering. Apart from grading and facilitating discusssion sessions, I helped students in reading papers, constructing summaries, and writing effective reviews.  Some of these students went on to become my co-authors in research publications later. 

During my tenure at University of California, Irvine, I've also had the opportunity to teach lectures when the instructors were either travelling or occupied with other engagements --- making it one of the most satisfying experiences I have ever had in my career. \\

\noindent \textbf{Advising:} As a senior Ph.D. student in the Mondego lab at University of California, Irvine, I had
the opportunity to mentor and advice a number of students on their thesis and research work.
More recently, I have had the pleasure of mentoring Ph.D. students during their internship at Tools for Software Engineers (TSE) group at Microsoft.
I have been able to successfully help them get their work published in top software engineering conferences and journals. 
This experience over the last few years has taught me the most ideal ways to bring out the best research capabilities among students. I have
been able to mentor the students on not only identifying solutions, but more importantly, identifying problems that need to be solved.

\section*{What I would like to teach?}

\begin{itemize}

\item I believe that software engineers today are less likely to design data structures and algorithms from scratch and more likely to build systems from library and framework components.
I'd like to design a course where the fundamental goal would be to engage students with concepts related to the construction of software systems at scale, building on their understanding of the basic building blocks of data structures, algorithms, program structures, and computer structures. 
As such, the course would cover technical topics in four areas: (1) concepts of design for complex systems, (2) object oriented programming, (3) static and dynamic analysis for programs, and (4) concurrent and distributed software.
The course would be designed to help students experience the real world requirements of software develoment, and use the tools that are commonly used in the industry to accomplish these tasks. The project would complement the theory taught in class.

\item I would like to teach a seminar course on software engineering research for graduate students where we would read and discuss seminal research papers in the area. This would give every software engineering student the basics needed. In addition, it would also enable them to experience a well written and rigorously researched idea. Such courses have had profound impact on my research during my graduate studies at Univesity of California, Irvine. I believe they inspire intellectual curiosity and critical thinking in everyone. 

\item Finally, I would like to develop a new course for early graduate students on ‘Empirical Methods’. Empirical methods play a key role in the evaluation of tools, technologies, and social and technical theories. Regardless of their research area, chances are that students will be conducing empirical studies as part of their research. This course would be a survey of empirical methods and provide an overview and hands on experience with a core of qualitative and quantitative empirical research methods, including interviews, qualitative coding, survey design, and large-scale mining and analysis of data. Students will mine and integrate data from and across online software repositories (e.g., GitHub and Stack Overflow) and employ a spectrum of data analysis techniques, ranging from statistical modeling to social network analysis. Ideally, I would invite my past collaborators, Tom Zimmermann and Nachiappan Nagappan from Microsoft Research, Prof. Cristina Lopes from University of California, Irvine, Prof. Margaret Anne Storey from University of Victoria, Canada who are experts in data analysis, to give guest lectures. Such a course would prepare the students for a whole new world of opportunities, beyond just software engineering companies.
\end{itemize}

I've been very fortunate to have mentors who have always challenged me to excel, foster collaborative discussions, develop critical thinking, and articulate and communicate my thoughts effectively. 
These lessons have remained with me and have profound impact in whatever I do.  Going forward, I would like to give back to my students what I've learned and in the process also learn from them.


\end{document}
