
%\documentstyle[12pt]{article}
%\pdfminorversion=4
\documentclass[a4paper, 12pt]{article}

\setlength{\oddsidemargin}{0in}
\setlength{\evensidemargin}{0in}
\setlength{\textwidth}{6.5in}
\setlength{\topmargin}{-.3in}
\setlength{\textheight}{9in}

\usepackage{hyperref}
\usepackage{lastpage}
%\pagestyle{empty}
\usepackage{fancyhdr}
\usepackage{pagecounting}
\usepackage[dvips]{color}
\definecolor{gray}{rgb}{0.4,0.4,0.4}
\thispagestyle{fancy}
\fancyfoot[C]{\small \textcolor{black}{Hitesh Sajnani -- }\thepage{}~\small{of}~\pageref{LastPage} -- \small{Diversity Statement}} 
\usepackage{graphicx}
\usepackage{svg}
\usepackage{csquotes}
%\usepackage[fontsize=11pt]{scrextend}
%\graphicspath{ {./images/} }
\begin{document}
%\usepackage{svg}

\begin{center}
{\LARGE \textbf{Diversity Statement}} \\[.3in]
{\large  \textbf{Hitesh Sajnani}} \\
{\small hsajnani@uci.edu}
\end{center}
\pagestyle{fancy}
\lhead{\textcolor{black}{\it Hitesh Sajnani}}
\rhead{\textcolor{black}{\thepage/\pageref{LastPage}}}


%\vspace*{.5in}


%My own journey has had many happier days, and from these too I have learned some important lessons. By high school, I found an amazing group of friends who I adore to this day. We were all very different. I was the unconventional math and yearbook “geek” who was also into soccer and the dance squad. In retrospect, with a strong sense of belonging, I was able to bring my full authentic self to school and that enabled me to thrive. I was lucky.


While growing up in a small town in India, I was always sourrounded by people of different cultures, languages, religions, and food preferences, yet all knitted together in a very close manner. I was fortunate to experience a similar confluence of cultures at University of California, Irvine, and specifically in Mondego lab where my labmates were from all over the world. Today, the tradition continues at Microsoft, where my team consists of people of diverse age, gender, ethnicity and background.  Having this privilege of being associated with a diverse set of people throughout my life, I recognize the desire to be included and to belong is universal. As I’ve also come to appreciate, the more I learn, the less I know. There are so many layers to each of us as individuals, what experiences shaped us, who we are, and what affects us.  I am committed to our growth mindset and applying that curiosity and humility to learn about how I can be even more inclusive in my behavior, a better ally to someone not like me, a mentor to someone new to the team, and more awakened to my own biases or ignorance. Each day I am inspired whenever I see people around me taking intentional action to be more inclusive. 

While I was in University of California, Irvine, starting from my second year (2011) I volunteered as a mentor in English Conversational Program (ECP). The ECP program provided international students the opportunity to practice and improve their conversational English with the help of English speaking facilitators. Students from all over the globe were paired in one-on-one conversations and learned about different cultural perspectives, intercultural communication skills, and recognize the value of our diverse campus community. In the process, I learned to understand people for whom they are rather than for whom we might like them to be and made some of my best friends in the ECP program. I did not have the opportunity to mentor interns from under-represented minorities in my lab at UC Irvine, but I had a wonderful experience being advised by Prof. Cristina V. Lopes during my Ph.D. Her philosophy was that diverse groups drive better performance by almost every measure --- more innovative thinking, better decision-making, better understanding of our issues, better ideas, solutions, and even products.  She is my role model for what an advisor should be. In my role as professor and mentor, I plan to reach out to talented students from under-represented minorities, and mentor them into becoming successful engineers and scientists.  

In my roles as student and teaching assistant at UC Irvine, I witnessed some of the best teachers creating healthy and inclusive learning environment while teaching. They made conscious effort to avoid implicit bias in the classroom. They would explicity encourage diverse set of students to speak so that the class room discussion would not be dominated by a few expressive students. They'd form diverse groups and conduct activities that'd foster collaboration and personal growth rather than intense head-to-head competition. I'd see them being very mindful of their language and tone so that it'd not disproportionately discourage members of underrepresented groups. As I look back, I find those gestures to be seemingly-simple yet highly effective, and in my role as a teacher in future, I plan to borrow those methods to build supportive learning environment that'd enable all the students to learn and share without fear.

Currently at Microsoft, my responsibilities include hosting Inclusive Dialogues for leaders to connect with diverse employees. In addition, I am a also member of the recuritment committee for hiring diverse candidates for Tools for Software Engineering (TSE) group. A conscious effort by the committee has increased the reprsesentation of female interns by 6\% in the last couple of years in TSE. As part of my engagement in the recruitment process, I've realized the importance of cultural awareness and how the lack of consideration and regard appears to be based in both arrogance and ignorance which often results in the unfavorable outcomes. Therefore, I want to organize culture aware job interview preparation workshops to provide training opportunities to a much broader student population. I'd invite women and underrepresented minorities in STEM to serve as mentors in these workshops.

Over a period of time, I've realized that we need diversity of thought, diversity of context, diversity of experiences, and an inclusive culture to do research that will resonate broadly. And yet our ability to harness that power of diversity to achieve our goal can only happen if we can truly bring our authentic self to work, be heard, be recognized, and do our best work. In spite of barriers often associated with race and culture, I have been able to effectively interact with people of amazing diversity, and be accepted. For me this has been and is a very humbling experience making me extremely conscious of the concept of building an inclusive culture for everyone to thrive. The phrase, \enquote{I go where I’m invited, but I stay where I’m welcomed} is a reminder that we must work on both diversity and inclusion if we are to make sustainable progress. 











\end{document}
