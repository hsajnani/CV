%-------------------------
% Resume in Latex
% Author : Sourabh Bajaj
% License : MIT
%------------------------

\documentclass[letterpaper,11pt]{article}

\usepackage{latexsym}
\usepackage[empty]{fullpage}
\usepackage{titlesec}
\usepackage{marvosym}
\usepackage[usenames,dvipsnames]{color}
\usepackage{verbatim}
\usepackage{enumitem}
\usepackage[hidelinks]{hyperref}
\usepackage{fancyhdr}
\usepackage[english]{babel}
\usepackage{url}
%\usepackage{marvosym} % For cool symbols.
\usepackage{hyperref}
\usepackage{lastpage}
\usepackage{fancyhdr}
\fancyhead{}
\fancyfoot{}
%\fancyhf{} % clear all header and footer fields
\cfoot{\footnotesize{Hitesh Sajnani -- }\thepage{}~\footnotesize{of}~\pageref{LastPage} -- \scriptsize{Curriculum Vitae}}
\pagestyle{fancy}
\renewcommand{\headrulewidth}{0pt}
\renewcommand{\footrulewidth}{0pt}

% Adjust margins
\addtolength{\oddsidemargin}{-0.5in}
\addtolength{\evensidemargin}{-0.5in}
\addtolength{\textwidth}{1in}
\addtolength{\topmargin}{-.5in}
\addtolength{\textheight}{1.0in}
\urlstyle{same}

\raggedbottom
\raggedright
\setlength{\tabcolsep}{0in}

% Sections formatting
\titleformat{\section}{
  \vspace{-4pt}\scshape\raggedright\large
}{}{0em}{}[\color{black}\titlerule \vspace{-5pt}]

%-------------------------
% Custom commands
\newcommand{\resumeItem}[2]{
  \item\small{
    \textbf{#1}{ #2 \vspace{-3pt}}
  }
}

\newcommand{\resumeSubheading}[5]{
  \vspace{-3pt}\item
    \begin{tabular*}{0.97\textwidth}[t]{l@{\extracolsep{\fill}}r}
      \textbf{#1} & #2 \\
      \textit{\small#3} & \textit{\small #4} \\
      \textit{\small#5}  
    \end{tabular*}\vspace{-12pt}
}

\newcommand{\resumeSubItem}[2]{\resumeItem{#1}{#2}\vspace{-4pt}}

\renewcommand{\labelitemii}{$\circ$}

\newcommand{\resumeSubHeadingListStart}{\begin{itemize}[leftmargin=*]}
\newcommand{\resumeSubHeadingListEnd}{\end{itemize}}
\newcommand{\resumeItemListStart}{\begin{itemize}}
\newcommand{\resumeItemListEnd}{\end{itemize}\vspace{-7pt}}

\newcommand{\cvreference}[7]{%
    \textbf{#1}\newline% Name
    \ifthenelse{\equal{#2}{}}{}{\addresssymbol~#2\newline}%
    \ifthenelse{\equal{#3}{}}{}{#3\newline}%
    \ifthenelse{\equal{#4}{}}{}{#4\newline}%
    \ifthenelse{\equal{#5}{}}{}{#5\newline}%
    \ifthenelse{\equal{#6}{}}{}{\emailsymbol~\texttt{#6}\newline}%
    \ifthenelse{\equal{#7}{}}{}{\phonesymbol~#7}}

%-------------------------------------------
%%%%%%  CV STARTS HERE  %%%%%%%%%%%%%%%%%%%%%%%%%%%%


\begin{document}

%----------HEADING-----------------
\begin{tabular*}{\textwidth}{l@{\extracolsep{\fill}}r}
  \textbf{\href{http://ics.uci.edu/~hsajnani}{\huge Hitesh Sajnani}} & Email : \href{mailto:hitsaj@microsoft.com}{hsajnani@uci.edu}\\
  \url{https://www.ics.uci.edu/~hsajnani/} & Mobile : +1-832-278-7005 \\  \\ \\ 
\end{tabular*}

%-----------RESEARCH INTEREST-----------------
\section{Research Interests}

%My primary line of research is in the field of software engineering with an emphasis on developing theoretical and practical tools and techniques for helping people to build, understand, and modify software systems.
%Using techniques from diverse domains including machine learning, information retrieval and statistical modeling, I analyze large scale software repositories to understand ongoing software engineering practices and build novel techniques tools to improve software quality and programmer productivity.

Empirical Software Engineering, Large Scale Code Clone Detection and Management, Distributed Build Systems, Developer Productivity, Software Maintenance and Evolution Metrics


%Open Source Software communities, Communication
%and Collaboration in Software Engineering, Code Review, Branching in Source Code
%Repositories, Release Engineering

%-----------EDUCATION-----------------
\section{Education}
  \resumeSubHeadingListStart
    \resumeSubheading
      {Ph.D., Information and Computer Science, University of California, Irvine}{2010 -- 2016}
      {Thesis topic: Large Scale Code Clone Detection} {}
      {Advisor: Prof. Cristina V. Lopes } % {   2010 -- 2016}
    \resumeSubheading
      {M.S., Information and Computer Science, University of California, Irvine}{2010 -- 2012}
      {Thesis topic: Enhancing Component Identification Using Machine Learning}{}
        {Advisor: Prof. Cristina V. Lopes }
    \resumeSubheading
      {B.Tech, Computer Engineering, Dharmsinh Desai University, India}{2003 -- 2007}
      {GPA: 4.00/4.00; Gold Medalist}{}
      {}
  \resumeSubHeadingListEnd

%-----------AWARDS-----------------

\section{Honors and Awards}
  \resumeSubHeadingListStart

 \resumeSubheading
  {Distinguished Artifact Award}{}
  {Object-Oriented Programming, Systems, Languages \& Applications (OOPSLA), 2017.} {}
  {}
  \resumeSubheading
  {Nomination for ACM SIGSOFT Outstanding Doctoral Dissertation Award}{}
  {Nominated by Department of Informatics, UC Irvine, 2017} {}
   {}
  \resumeSubheading
  {Distinguished Paper Award}{}
  {International Conference on Software Maintenance and Evolution (ICSME), 2016.} {}
  {}
  \resumeSubheading
  {ACM Student Scholar Award}{}
  {ACM Turing Centenary Celebration, San Francisco, 2013.} {}
  {}
  \resumeSubheading
  {Best Industrial Paper Award}{}
  {Working Conference on Reverse Engineering (WCRE), 2011.} {}
{}
  \resumeSubheading
  {Outstanding Paper Award}{}
  {Corporate Technology Board Conference, Tata Consultancy Service, 2011.} {}
  {}
  \resumeSubheading
  {Eklavya Award (equivalent to \emph{summa cum laude)} }{}
  {Awarded for securing the highest cumulative GPA among all engineering students across all disciplines, India, 2007. } {}
  {}
   \resumeSubHeadingListEnd

\section{Patents}

  \resumeSubHeadingListStart
    \resumeSubheading
      {Source Code File Recommendation Notification}{}
      {Patent\# 404762-US-NP}{Aug. 2018}
      {} {\small{Automatic recommendation of potentially missed files during a pull request creation to complete a change based on historical mining and modeling of pull requests' metadata.}}

\newpage

   \resumeSubHeadingListEnd
%-----------EXPERIENCE-----------------
\section{Research \& Work Experience}
  \vspace{5pt}
  \resumeSubHeadingListStart

    \resumeSubheading
      {Microsoft, Tools for Software Engineers (TSE) group}{Redmond, WA}
      {Research Engineer}{Mar 2016 - Present}
      {}
     \vspace{-15pt}
      \resumeItemListStart
          \resumeItem {} {Analyzing factors influencing developer satisfaction and productivity.}
          \resumeItem {} {Building cutting edge tools and technology to assist software engineers build better software.}
      \resumeItemListEnd



     \resumeSubheading
      {University of California, Irvine}{Irvine, CA}
      {Graduate Student Researcher}{2010 -- 2016}
     {}
     \vspace{-15pt}
      \resumeItemListStart
        \resumeItem{}
          {Advanced research in tools and techniques for scalable code clone detection. (Ph.D. thesis topic)}
        \resumeItem{}
          {Developed machine learning based approaches to software architecture recovery. (Master’s thesis topic)}
        \resumeItem{}
          {Mined and analyzed data from software repositories to gain insights about software evolution.}
       \resumeItemListEnd

\vspace{5pt}
    \resumeSubheading
      {Microsoft Research, Mentor: Dr. Rob DeLine \& Dr. Wolfram Schulte}{Redmond, WA}
      {Research Intern}{Summer 2013}
     {}
     \vspace{-15pt}
      \resumeItemListStart
        \resumeItem {} {Analyzed software change data of products like Office, Bing, and Windows to understand and
improve their development practices at Microsoft.}
        \resumeItem {} {Helped design and evaluate \emph{Tempe}, a research prototype for interactive and collaborative webservice for data analysis.}
      \resumeItemListEnd

\vspace{5pt}
    \resumeSubheading
      {Microsoft Research, Mentor: Dr. Owen Lyold}{Mountain View, CA}
      {Research Intern}{Summer 2012}
     {}
     \vspace{-15pt}
      \resumeItemListStart
        \resumeItem {} {Designed and developed a machine learning based entity resolution algorithm and tool for detecting similar short text records.}
\resumeItemListEnd
 
\vspace{5pt}
  \resumeSubheading
      {SRCH2, Mentor: Dr. Chen Li}{Irvine, CA}
      {Research Intern}{Summer 2011}
     {}
     \vspace{-15pt}
      \resumeItemListStart
        \resumeItem {} {SRCH2 is a research group at UC Irvine specializing in powerful search technology. I worked on developing state-of-the-art algorithms for location based search engine that included features like instant and fuzzy search.}
\resumeItemListEnd

\vspace{5pt}
  \resumeSubheading
      {Tata Research Development and Design Center}{Pune, India}
      {Researcher}{2007 -- 2010}
      {}
     \vspace{-15pt}
      \resumeItemListStart
        \resumeItem {} {Facilitated software evolution of business systems by improving their architecture through re-factoring, code partitioning, layering, and pattern enforcement.}
      \resumeItemListEnd

\vspace{5pt}
  \resumeSubheading
      {Tata Research Development and Design Center, Mentors: Ravindra Naik}{Pune, India}
      {Research Intern}{Winter 2007}
       {}
     \vspace{-15pt}
      \resumeItemListStart
        \resumeItem {} {Developed Software Healthmeter, a metrics (structural and behavioral) based research prototype to determine software maintainability.}
      \resumeItemListEnd

  \resumeSubHeadingListEnd


\section{Teaching Experience}
  \resumeSubHeadingListStart

    \resumeSubheading
      {\small{Informatics 191 A: Senior Design Project:  Usability Engineering \& Software Development }}{\small{UC Irvine}}
      {Teaching Assistant;  Level: Undergrad; Class Size: 80 students}{Spring 2011}
      {} {Group supervised project in which students analyze, specify, design, construct, evaluate and adapt a significant information processing system.}

    \resumeSubheading
      {\small{Informatics 43: Introduction to Software Engineering }}{\small{UC Irvine}}
      {Teaching Assistant; Level: Grad; Class Size: 40 students}{Winter 2011}
      {} {Concepts, methods, and current practice of software engineering. Large-scale software production, software life cycle models, principles and techniques for each stage of development.}

    \resumeSubheading
      {\small{Informatics 133: User Interaction Software}}{\small{UC Irvine}}
      {Teaching Assistant;  Level: Undergrad; Class Size: 50 students}{Fall 2010}
      {} {Introduction to human-computer interaction programming. Emphasis on current tools, standards, methodologies for implementing effective interaction designs. Strategies for evaluation of user interfaces}

%\resumeSubHeadingListStart
  \resumeSubHeadingListEnd

\newpage

%-----------JOURNAL PUBLICATIONS-----------------

\section{Refereed Publications}
\vspace{5pt}
\textbf{Journal papers}

\begin{enumerate}

 \item\small{Vaibhav Saini, \textbf{Hitesh Sajnani}, and Cristina Lopes. “An Empirical Analysis on the
Relationship Between Software Clones and Quality Metrics,” Empirical Software Engineering Journal, June 2017 \textbf{[Invited Paper]}}

 \vspace{-1pt}\item\small{\textbf{Hitesh Sajnani}, Vaibhav Saini, and Cristina Lopes. “A Parallel and Efficient Approach
to Large Scale Code Clone Detection,” Journal of Software: Evolution and Process, March 2015 \textbf{[Invited Paper]}}

\end{enumerate}



\textbf{ Full Conference Papers}

\begin{enumerate}

\item\small{Pavneet Kochhar, Stanislaw Swierc, Trevor Carnahan, \textbf{Hitesh Sajnani}, Meiyappan Nagappan. “Understanding the role of reporting in work item tracking systems for software development: an industrial case study,” in Proceedings of International Conference on Software Maintenance and Evolution, Sept. 2018, Madrid, Spain, USA  }

 \vspace{-1pt}\item\small{Cristina Lopes, Petr Maj, Pedro Martins, Vaibhav Saini, Di Yang, Jakub Zitny, \textbf{Hitesh Sajnani}, and Jan Vitek. “DéjàVu: a map of code duplicates on GitHub,” in Proceedings of ACM on Programming Languages, OOPSLA, Oct. 2017, Vancouver, Canada \textbf{[Distinguished Artifact Award]}  }

 \vspace{-1pt}\item\small{Adriano de Paula, Eduardo Guerra, \textbf{Hitesh Sajnani}, Cristina Lopes and Otavio Lemos. “An Exploratory Study of Interface Redundancy in Code Repositories,” in Proceedings of Working Conference on Source Code Analysis and Manipulation, Oct. 2016, Raleigh, USA  }

 \vspace{-1pt}\item\small{Vaibhav Saini, \textbf{Hitesh Sajnani}, and Cristina Lopes. “Comparing Quality Metrics for
Cloned and Non-Cloned Java Methods: A Large Scale Empirical Study,” in Proceedings
of International Conference on Software Maintenance and Evolution, Oct. 2016,
Raleigh, USA \textbf{[Distinguished Paper Award]}}
   \vspace{-1pt}\item\small{\textbf{Hitesh Sajnani}, Vaibhav Saini, Jeffrey Svejlanko, Chanchal Roy and Cristina Lopes.
“SoucererCC: Scaling Token-Based Code Clone Detection to Big-Code,” in Proceedings
of International Conference on Software Engineering, May 2016, Austin, USA  }
   \vspace{-1pt}\item\small{Otavio Lemos, Adriano de Paula, \textbf{Hitesh Sajnani}, and Cristina Lopes. “Can the Use of Types
and Query Expansion Help Improve Large-Scale Code Search?,” in Proceedings of
Working Conference on Source Code Analysis and Manipulation, Sept. 2015, Bremen,
Germany  }
   \vspace{-1pt}\item\small{\textbf{Hitesh Sajnani}, Vaibhav Saini, and Cristina Lopes. “A Comparative Study of Bug
Patterns in Java Cloned and Non-cloned Code,” in Proceedings of Working Conference
on Source Code Analysis and Manipulation, Sept. 2014, Victoria, Canada }
   \vspace{-1pt}\item\small{\textbf{Hitesh Sajnani}, Vaibhav Saini, and Cristina Lopes. “Is Popularity a Measure of Quality?
An Analysis of Maven Components,” in Proceedings of the International Conference on
Software Maintenance and Evolution, Sept. 2014, Victoria, Canada }
   \vspace{-1pt}\item\small{Lukas Schulte, \textbf{Hitesh Sajnani}, and Jacek Czerwonka. “Active Files as a Measure of
Software Maintainability,” in Proceedings of the International Conference on Software
Engineering, June 2014, Hyderabad, India }
   \vspace{-1pt}\item\small{\textbf{Hitesh Sajnani} and Cristina Lopes. “Probabilistic Component Identification,” in Proceedings
of the India Software Engineering Conference, Feb 2014, Chennai, India }
   \vspace{-1pt}\item\small{Joel Ossher, \textbf{Hitesh Sajnani}, and Cristina Lopes. “ASTRA: Bottom-up Construction
of Structured Artifact Repositories,” Proceedings of Working Conference on Reverse
Engineering, Oct. 2012, Kingston, Canada }
 \vspace{-1pt}\item\small{\textbf{Hitesh Sajnani}, Ravindra Naik and Cristina Lopes. “Easing Software Evolution: A
Change-data and Domain-driven Approach,” in Proceedings of India Software Engineering
Conference, Feb 2012, Kanpur, India }
 \vspace{-1pt}\item\small{\textbf{Hitesh Sajnani}, Ravindra Naik, and Cristina Lopes. “Application Architecture Discovery:
Towards Domain-driven, Easily Extensible Code Structure,” in Proceedings of
Working Conference on Reverse Engineering, Oct. 2011, Limerick, Ireland \textbf{[Best Industrial
Paper Award], [Outstanding Paper Award, Tata Consultancy Services]} }
 \vspace{-1pt}\item\small{Joel Ossher, \textbf{Hitesh Sajnani}, and Cristina Lopes. “ Clone Detection in Open Source
Java Projects: The Good, The Bad, and The Ugly,” in Proceedings of International
Conference on Software Maintenance, Sept. 2011, Williamsburg, USA }

\end{enumerate}
\textbf{Workshop, Tool, and Other Short Papers}

\begin{enumerate}

\item\small{Pavneet Kochhar, Stanislaw Swierc, Trevor Carnahan, \textbf{Hitesh Sajnani}, Meiyappan Nagappan. “Understanding the role of reporting in work item tracking systems for software development: an industrial case study,” in Proceedings of International Conference on Software Engineering, May 2018, Gothenburg, Sweden  }

   \vspace{-1pt}\item\small{Vaibhav Saini, \textbf{Hitesh Sajnani}, Jaewoo Kim and Cristina Lopes. “SourcererCC and
SourcererCC-I: Tools to Detect Clones in Batch mode and During Software Development,”
in Proceedings of International Conference on Software Engineering, May 2016,
Austin, USA }

   \vspace{-1pt}\item\small{Vaibhav Saini, \textbf{Hitesh Sajnani}, and Cristina Lopes. “A Dataset for Maven Artifacts and
Bug Patterns Found in Them,” in Proceedings of the Mining Software Repositories, June
2014, Hyderabad, India }
   \vspace{-1pt}\item\small{Vaibhav Saini, \textbf{Hitesh Sajnani}, Jaewoo Kim, and Cristina Lopes. “SourcererCC and
SourcererCC-I: Tools to Detect Clones in Batch mode and During Software Development,”
in Proceedings of International Conference on Software Engineering, May 2016,
Austin, USA }
   \vspace{-1pt}\item\small{\textbf{Hitesh Sajnani} and Cristina Lopes. “A Parallel and Efficient Approach to Large Scale
Code Clone Detection,” in Proceedings of International Workshop on Software Clones,
May 2013, San Francisco, USA }

   \vspace{-1pt}\item\small{\textbf{Hitesh Sajnani}. “Automatic Software Architecture Recovery: A Machine Learning
Approach,” in Proceedings of International Conference on Program Comprehension, June
2012, Passau, Germany }

   \vspace{-1pt}\item\small{\textbf{Hitesh Sajnani} and Cristina Lopes. “A Parallel and Efficient Approach to Large Scale
Code Clone Detection,” in Proceedings of International Workshop on Software Clones,
May 2013, San Francisco, USA }
   \vspace{-1pt}\item\small{\textbf{Hitesh Sajnani}, Sarah Javanmardi, David McDonald, and Cristina Lopes. “Multi-Label
Classification of Short Text: A Casestudy on Wikipedia Barnstars,” in Proceedings
of the AAAI Conference on Analyzing Microtext, Aug. 2011, San Francisco, USA}
   \vspace{-1pt}\item\small{Lee Martie, Vijay Krishna Palepu, \textbf{Hitesh Sajnani}, and Cristina Lopes. “Trendy Bugs: Topic
Trends in the Android Bug Reports,” in Proceedings of Mining Software Repositories,
June 2012, Zurich, Switzerland }
   \vspace{-1pt}\item\small{\textbf{Hitesh Sajnani}, Joel Ossher, and Cristina Lopes. “Parallel Code Clone Detection Using
MapReduce,” in Proceedings of International Conference on Program Comprehension, June
2012, Passau, Germany }
   \vspace{-1pt}\item\small{\textbf{Hitesh Sajnani}, Joel Ossher, and Cristina Lopes. “Parallel Code Clone Detection Using
MapReduce,” in Proceedings of International Conference on Program Comprehension, June
2012, Passau, Germany }
   \vspace{-1pt}\item\small{Ravindra Naik and \textbf{Hitesh Sajnani}. “Using Change History of Software To Improve
Software Evolvability,” in Proceedings of India Software Engineering Conference, Feb 2010, Mysore, India}

\end{enumerate}


\textbf{Papers under Review}

\begin{enumerate}
 \item\small{Michele Tufano, \textbf{Hitesh Sajnani}, Kim Herzig. “Towards Predicting the Impact of Software Changes on Building Activities,” in International Conference on Software Engineering, May 2019, Montreal, Canada}

   \vspace{-1pt}\item\small{Vaibhav Saini, Farima Farmahinifarahani, Yadong Lu, Di Yang, Pedro Martins, \textbf{Hitesh Sajnani}, Pierre Baldi, and Cristina Lopes. “Towards Automating Precision Studies of Clone Detectors,” in International Conference on Software Engineering, May 2019, Montreal, Canada}

   \vspace{-1pt}\item\small{Farima Farmahinifarahani, Vaibhav Saini, Di Yang, \textbf{Hitesh Sajnani}, and Cristina Lopes. “On Precision of Code Clone Detection Tools,” in International Conference on Software Analysis, Evolution and Reengineering, Feb. 2019, Hangzhou, China}

\end{enumerate}

\newpage

\section{Professional Services}

\textbf{Organization Committee Member}

\begin{itemize}

  \item\small{Co-Chair, Program, International Working Conference on Source Code Analysis and Manipulation, Engineering Track, SCAM’19, Cleveland, USA }
  \vspace{-5pt}\item\small{Co-Chair, Program, International Conference on Program Comprehension, Industry Track, ICPC’18, Gothenburg, Sweden }
   \vspace{-5pt}\item\small{Co-Chair, Program, International Workshop on Software Clones, IWSC’17, Klagenfurt, Austria }
   \vspace{-5pt}\item\small{Chair, Social Media, International Conference on Software Maintenance and Evolution, ICSME’15, Bremen, Germany }
  \vspace{-5pt}\item\small{Co-chair, Local arrangements, International Conference on Program Comprehension, ICPC’14, Hyderabad, India }
\end{itemize}

\textbf{Program Committee Member}


\begin{itemize}

  \item\small{International Conference on Software Maintenance 2016, 2017, 2018}
  \vspace{-5pt}\item\small{International Workshop on Software Clones 2015, 2016, 2018, 2019}
  \vspace{-5pt}\item\small{International Conference on Program Comprehension 2017, 2018}
  \vspace{-5pt}\item\small{ International Working Conference on Source Code Analysis and Manipulation 2017, 2018}
  \vspace{-5pt}\item\small{ International Workshop on Software Analytics 2016, 2017}
   \vspace{-5pt}\item\small{International Conference on Mining Software Repositories, Data Track, 2016}

\end{itemize}

\textbf{Artifact Evaluation Committee Member}

\begin{itemize}
\item\small{SIGPLAN conference on Systems, Programming, Languages and Applications, SPLASH 2013,
Indianapolis, USA}
  \vspace{-5pt}\item\small{International Conference on Software Engineering, ICSE 2019,
Montreal, Canada}

\end{itemize}

\textbf{Journal Reviewer}

\begin{itemize}

  \item\small{IEEE Transactions on Software Engineering 2016, 2017, 2018}
  \vspace{-5pt}\item\small{Empirical Software Engineering 2016, 2017, 2018}
  \vspace{-5pt}\item\small{Journal of Systems and Software 2018}
  \vspace{-5pt}\item\small{Science of Computer Programming 2018}
  \vspace{-5pt}\item\small{Journal on Software Quality 2014, 2015}
  \vspace{-5pt}\item\small{Journal for Software: Evolution and Processes 2013}

\end{itemize}

\textbf{Student Volunteer}
\begin{itemize}
\item\small{International Conference on Software Engineering, ICSE 2014,
Hyderabad, India}

  \vspace{-5pt}\item\small{ACM Turing Award Centenary Celebration, 2012, San Francisco, USA}

  \vspace{-5pt}\item\small{SIGPLAN conference on Systems, Programming, Languages and Applications, SPLASH 2011,
Portland, USA}
\end{itemize}

\newpage

\section{Public Research Talks \& Presentations}
\vspace{5pt}
\textbf{Invited Talks}
\begin{itemize}

 \item\small{Large Scale Code Clone Detection at Tools for Software Engineers Group, Microsoft, 2016, Redmond, USA}

  \vspace{0pt}\item\small{Web Search \& Indexing, invited lecture in the class of Information Retrieval at UC Irvine, 2016, Irvine, USA}

  \vspace{0pt}\item\small{Web Crawling, invited lecture in the class of Information Retrieval at UC Irvine, 2015, Irvine, USA}

  \vspace{0pt}\item\small{Using Tempe for Empirical Software Engineering Studies, Microsoft Research, 2013, Redmond, USA}

  \vspace{0pt}\item\small{Large Scale Code Clone Detection, Tata Research Development and Design Center, 2013, Pune, India}

  \vspace{0pt}\item\small{Latent Dirichlet Allocation: Uses in SE Research, Tata Research Development and Design Center, 2013,  Pune, India}

  \vspace{0pt}\item\small{Software Architecture Recovery, invited lecture in the class of Software Engineering at UC Irvine, 2011, Irvine, USA}

\end{itemize}

\textbf{Conference Presentations}
\begin{itemize}

  \item\small{Understanding the role of reporting in work item tracking systems for software development: an industrial case study, in International Conference on Software Maintenance and Evolution, 2018, Madrid, Spain}

  \vspace{0pt}\item\small{Comparing Quality Metrics for Cloned and Non-Cloned Java Methods: A Large Scale Empirical Study, in International Conference on Software Maintenance and Evolution, 2016, North Carolina, USA}

  \vspace{0pt}\item\small{SoucererCC: Scaling Token-Based Code Clone Detection to Big-Code, in International Conference on Software Engineering, 2016, Texas, USA}

  \vspace{0pt}\item\small{Is Popularity a Measure of Quality? An Analysis of Maven Components, in International Conference on Software Maintenance and Evolution, 2014, Victoria, Canada}

  \vspace{0pt}\item\small{A Dataset for Maven Artifacts and Bug Patterns Found in Them, in Working Conference on Mining Software Repositories, 2014, Hyderabad, India}

  \vspace{0pt}\item\small{Active Files as a Measure of Software Maintainability, in International Conference on Software Engineering, 2014, Hyderabad, India}

  \vspace{0pt}\item\small{Probabilistic Component Identification, in India Software Engineering Conference, 2014, Chennai, India}

  \vspace{0pt}\item\small{A Parallel and Efficient Approach to Large Scale Code Clone Detection, in International Workshop on Software Clones, 2013, San Francisco, USA}

  \vspace{0pt}\item\small{Automatic Software Architecture Recovery: A Machine Learning Approach, in International Conference on Program Comprehension, 2012, Passau, Germany}

  \vspace{0pt}\item\small{Easing Software Evolution: A Change-data and Domain-driven Approach, in India Software Engineering Conference, 2012, Kanpur, India}

  \vspace{0pt}\item\small{Multi-Label Classification of Short Text: A Casestudy on Wikipedia Barnstars, in Workshop on Analyzing Microtext, AAAI, 2011, San Francisco, USA}

  \vspace{0pt}\item\small{Application Architecture Discovery: Towards Domain-driven, Easily Extensible Code Structure, in Working Conference on Reverse Engineering, 2011, Limerick, Ireland}

  \vspace{0pt}\item\small{Using Change History of Software To Improve Software Evolvability, in India Software Engineering Conference, 2010, Mysore, India}

\end{itemize}
\newpage

\section{Research Students Supervised}


\begin{itemize}

\vspace{5pt}\item\small{Michele Tufano, Ph.D. Student, College of Williams and Mary, Williamsburg, Virginia; Research Intern at Tools for Software Engineers group, Microsoft, Summer 2018.} \\
\vspace{0pt}\item\small{Pavneet Kochhar, Ph.D. Student Singapore Management University, Singapore; Research Intern, Tools for Software Engineers group, Microsoft, Summer 2017. (Pavneet is now a software engineer at Microsoft Azure.)} \\
\vspace{0pt}\item\small{Vaibhav Saini, Ph.D. Student  University of California, Irvine; Research Intern, Tools for Software Engineers group, Microsoft, Summer 2016.} \\
\vspace{0pt}\item\small{Caio Lopes, Bachelor Engineering,  University de Sao Paulo, Brazil; Research Intern, University of California, Irvine, Summer 2015 (co-mentored w/ Prof. Lopes. Caio is now a data scientist at Itau Unibanco)}  \\
\vspace{0pt}\item\small{Jaewoo Kim, M.S. Student, University of California, Irvine; Individual Study (co-mentored w/ Prof. Lopes. Jaewoo is now a software engineer at 9M Interactive)} \\ 
\vspace{0pt}\item\small{Lukas Schulte, B.S. Boston University, Research Intern, Tools for Software Engineers group, Microsoft, Summer 2013  (Lukas is now a director of engineering at OCTI, INC)}\\
\vspace{0pt}\item\small{Pramit Chaudhary, M.S. Student, University of California, Irvine, Individual Study (co-mentored w/ Prof. Lopes. Pramit is now a data scientist at e-Harmony)} \\
\vspace{0pt}\item\small{Praneet Mhatre, M.S. Student, University of California, Irvine, Individual Study  (co-mentored w/ Prof. Lopes. Praneet is now a data scientist at Zillow)} \\
\vspace{0pt}\item\small{Prashil Gupta, B.S. Student, Dharmsinh Desai Univesity, India;  Research Intern, Tata Research Development and Design Center, Pune, India, Summer 2009. ( Prashil is now a software engineer at Cisco )} \\
\vspace{0pt}\item\small{Dharam Thacker, B.S. Student, Dharmsinh Desai Univesity, India;  Research Intern, Tata Research Development and Design Center, Pune, India, Summer 2009. ( Dharam is now an architect at JP Morgan ) }\\
\vspace{0pt}\item\small{Manish Agarwal, B.S. Student, NIT Warangal, India, Research Intern, Tata Research Development and Design Center, Pune, India, Summer 2009. ( Manish is now a software engineer at Intel )} \\

\end{itemize}

\section{References}

\begin{tabular}{lr}
\begin{minipage}[t]{3.5in}
Dr.\ Cristina V. Lopes\\
Professor, Department of Informatics \\
Donald Bren School of Information and \\
Computer Sciences\\
University of California\\
Irvine, CA \\
lopes@uci.edu \\
\end{minipage}
&
\begin{minipage}[t]{3.5in}
Dr.\ Nachiappan Nagappan\\
Principal Researcher\\
Microsoft Research\\
Empirical Software Engineering Group (ESE)\\
Redmond, WA\\
nachin@microsoft.com\\
\end{minipage}
\\
\\ 
\begin{minipage}[t]{3.5in}
Dr.\ Thomas Zimmermann\\
Senior Researcher\\
Microsoft Research\\
Empirical Software Engineering Group (ESE)\\
Redmond, WA\\
tzimmer@microsoft.com\\
\end{minipage}
&
\begin{minipage}[t]{3.5in}
Dr.\ Michael W. Godfrey \\
Professor, David R. Cheriton School of Computer Science\\
University of Waterloo\\
Waterloo, Ontario, N2L 3G1, Canada \\
migod@uwaterloo.ca \\
\end{minipage}
\end{tabular}








%
%--------PROGRAMMING SKILLS------------
%\section{Programming Skills}
%  \resumeSubHeadingListStart
%    \item{
%      \textbf{Languages}{: Scala, Python, Javascript, C++, SQL, Java}
%      \hfill
%      \textbf{Technologies}{: AWS, Play, React, Kafka, GCE}
%    }
%  \resumeSubHeadingListEnd


%-------------------------------------------
\end{document}
